%% Example of use of pythontex in combination with pyCost
\documentclass[11pt]{article}% 
\usepackage{amsmath} 
\usepackage{pythontex} 
\usepackage{nopageno} 
\begin{document} 
 
 
\begin{pycode} 
 
import os
from pycost.structure import obra

obra= obra.Obra(cod="test", tit="Test title")

# Read data from file.
pth= os.path.dirname(__file__)
# print("pth= ", pth)
if(not pth):
    pth= '.'
obra.readFromYaml(pth+'/../../../verif/tests/data/test_file_05.yaml')

def getPriceDescription(code):
    unitPrice= obra.findPrice(code)
    return unitPrice.long_description

def getPriceTable(codes):
    tabData= list()
    for code in codes:
        description= getPriceDescription(code)
        tabData.append([code, description])
    retval= str()
    for row in tabData:
        retval+= row[0] + ' & ' + row[1] + '\\\\\n'
        retval+= '\\hline\n'
    return retval

\end{pycode}
 
The price table is:

\begin{center}
\begin{small}
  \begin{tabular}{|c|p{11cm}|}
    \hline
    Código & Descripción \\
    \hline
  \py{getPriceTable(['CANALACO36','CELAUS'])}
  \end{tabular}
  \end{small}
\end{center}
\end{document}
